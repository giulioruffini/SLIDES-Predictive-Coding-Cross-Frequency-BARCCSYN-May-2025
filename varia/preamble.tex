\usefonttheme{serif}

\usepackage[utf8]{inputenc}
\usepackage[T1]{fontenc}
\usepackage{lmodern}
\usepackage[super]{natbib}
%\usepackage[superscript]{cite}
\usepackage[english]{babel}
\usepackage{booktabs}
\usepackage{amsmath,amsthm}
\usepackage{tikz}
\usepackage{background}
\usepackage[overlay,absolute]{textpos} % couvertures

  \usepackage{changepage}                % page licence
  \usepackage{tabularx}                  % page licence
  \usepackage{booktabs}                  % beaux tableaux
  \usepackage{fontawesome5}              % icônes
  \usepackage{awesomebox}                % \tipbox et autres
  \usepackage{listings}                  % code source
  \usepackage[export]{adjustbox}         % cadre autour image

% \usepackage{bidicontour} % Contour with xelatex
% \usepackage{bidi}
% \bidicontourlength{0.1pt}
% % \usepackage{efbox,graphicx}


  

  
  
 %% Couleurs
  \definecolor{comments}{rgb}{0.5,0.55,0.6} % commentaires
  \definecolor{link}{rgb}{0,0.4,0.6}   % liens internes
  \definecolor{url}{rgb}{0.6,0,0}      % liens externes
  \definecolor{rouge}{rgb}{0.9,0,0.1}  % bandeau rouge UL
  \definecolor{or}{rgb}{1,0.8,0}       % bandeau or UL

  %% Hyperliens
  \hypersetup{%
    pdfauthor = {Olivier Côté},
    pdftitle = {Discrimination en assurance générale},
    colorlinks = false,
    linktocpage = true,
    urlcolor = {url},
    linkcolor = {link},
    citecolor = {citation},
    pdfpagemode = {UseOutlines},
    pdfstartview = {Fit}}
  \setlength{\XeTeXLinkMargin}{1pt}
  
  %% Sections de code source
  \lstloadlanguages{[LaTeX]TeX}
  \lstset{language=[LaTeX]TeX,
    escapeinside=`',
    extendedchars=true,
    inputencoding=utf8/latin1,
    basicstyle=\small\ttfamily\NoAutoSpacing,
    commentstyle=\color{comments},
    keywordstyle=\mdseries,
    emphstyle=\color{alert}\bfseries,
    backgroundcolor=\color{LightYellow1},
    frame=lr, rulecolor=\color{LightYellow1},
    showstringspaces=false}

  %%% =========================
  %%%  Nouveaux environnements
  %%% =========================

  %% Environnements pour les demo de code; tirés du document
  %% principal. (L'environnement 'eqxample' ajoute des filets de part
  %% et d'autre du bloc pour illustrer les marges.)
  \newenvironment{demo}{%
    \begin{beamercolorbox}[wd=\linewidth,sep=6pt]{block body example}}
    {\end{beamercolorbox}}
  \newenvironment{texample}[1][0.45\linewidth]{%
    \noindent\begin{minipage}{#1}%
      \def\producing{\end{minipage}\hfill\begin{minipage}{\dimexpr0.9\linewidth-#1}%
        \hbox\bgroup\kern-.2pt%
        \vbox\bgroup\parindent0pt\relax
        % The 3pt is to cancel the -\lineskip from \displ@y
        \abovedisplayskip3pt \abovedisplayshortskip\abovedisplayskip
        \belowdisplayskip0pt \belowdisplayshortskip\belowdisplayskip
        \noindent}
    }{%
      \par
      % Ensure that a lonely \[\] structure doesn't take up width less than
      % \hsize.
      \hrule height0pt width\hsize
      \egroup\kern-.2pt\egroup
    \end{minipage}%
    \par
  }
  \newenvironment{eqxample}{%
    \noindent\begin{minipage}{.45\linewidth}%
      \def\producing{\end{minipage}\hfill\begin{minipage}{.45\linewidth}%
        \hbox\bgroup\kern-.2pt\vrule width.2pt%
        \vbox\bgroup\parindent0pt\relax
        % The 3pt is to cancel the -\lineskip from \displ@y
        \abovedisplayskip3pt \abovedisplayshortskip\abovedisplayskip
        \belowdisplayskip0pt \belowdisplayshortskip\belowdisplayskip
        \noindent}
    }{%
      \par
      % Ensure that a lonely \[\] structure doesn't take up width less than
      % \hsize.
      \hrule height0pt width\hsize
      \egroup\vrule width.2pt\kern-.2pt\egroup
    \end{minipage}%
    \par
  }

  %% Simplfication de l'environnement 'quote' de beamer
  \renewenvironment{quote}{%
    \begin{beamercolorbox}[wd=\linewidth,sep=6pt]{block body example}}
    {\end{beamercolorbox}}

  %% Exercices
  \newenvironment{exercice}{%
    \begin{frame}[fragile=singleslide]
      \frametitle{\faCogs\; Exercice}}{\end{frame}}

  %% =====================
  %%  Nouvelles commandes
  %% =====================

  %% Noms de fonctions, code, environnement, etc.
  \newcommand{\code}[1]{\textcolor{code}{\texttt{#1}}}
  \newcommand{\fichier}[1]{\code{#1}}
  \newcommand{\class}[1]{\textbf{#1}}
  \newcommand{\pkg}[1]{\textbf{#1}}
  \newcommand{\link}[2]{\href{#1}{#2~\raisebox{-0.2ex}{\faExternalLink*}}}

  %% Pour documenter des commandes LaTeX; dérivé de memoir.cls
  \def\bs{\code{\char`\\}}
  \newcommand{\meta}[1]{%
    \ensuremath\langle{\normalfont\itshape #1\/}\ensuremath\rangle}
  \newcommand{\marg}[1]{%
    {\ttfamily\char`\{}\meta{#1}{\ttfamily\char`\}}}
  \newcommand{\oarg}[1]{%
    {\ttfamily\char`\[}\meta{#1}{\ttfamily\char`\]}}
  \newcommand{\cs}[1]{\code{\char`\\#1}}
  
    %%% =======
  %%%  Varia
  %%% =======

  %% Longueurs pour la composition des pages couvertures avant et
  %% arrière.
  \newlength{\banderougewidth} \newlength{\banderougeheight}
  \newlength{\bandeorwidth}    \newlength{\bandeorheight}
  \newlength{\imageheight}
  \newlength{\logoheight}

\usepackage{fontspec}
\usepackage{unicode-math}
\usefonttheme{serif}    % Pour avoir un "thème de police"

%%% On peut mettre la police principale ici


% \usepackage{arevmath}

% %%% Green checkmark
% \newcommand{\bluecheck}{}%
% \DeclareRobustCommand{\greencheck}{%
%   \tikz\fill[scale=0.4, color=black!30!green]
%   (0,.35) -- (.25,0) -- (1,.7) -- (.25,.15) -- cycle;%
% }

% %% To remove some annoying errors
% \pdfstringdefDisableCommands{%
%   \def\\{}%
%   \def\texttt#1{<#1>}%
% }

%% Pour les footnote dans un tableau
% \usepackage{footnote}
% \makesavenoteenv{tabular}
% \makesavenoteenv{table}
 
% \efboxsetup{linecolor=black!50,linewidth=0.5pt}
%\usepackage{adjustbox,lipsum}
%\frenchbsetup{CompactItemize=false}


% \usepackage[babel]{csquotes}
%\usepackage[url=false, doi=false, style=science, backend=bibtex, bibencoding=ascii]{biblatex}
% \bibliography{IEEEabrv,bib/OAM}

  
%\graphicspath{{img/}}


\mode<presentation> {
 %\useoutertheme{infolines} % Pour les thèmes qui n'ont pas de pied-de-page
 \usetheme{ulaval}
 %\usecolortheme{ulaval}
  \setbeamercovered{transparent}
 %\setbeamercovered{invisible}
 \setbeamertemplate{navigation symbols}{} % Enlever les icônes de navigation
}


%\usepackage{bm} 
% For typesetting bold math (not \mathbold)
%\includeonlyframes{}

%\usepackage{pgfpages}
%\setbeameroption{show notes on second screen}
%\setbeameroption{show notes}
%\setbeamertemplate{note page}[plain]

% \logo{%\includegraphics[height=0.6cm]{COPL}\hspace{.5cm}%
%  \includegraphics[height=0.6cm]{UL_P}\hspace{.2cm}\vspace{-.05\paperheight}}
 


%\date{24 janvier 2022}%\today} % \today will show current date. 
% Alternatively, you can specify a date.
%\today 



\newenvironment{xframe}[2][]
  {\begin{frame}[fragile,environment=xframe,#1]
  \frametitle{#2}}
  {\end{frame}}

\AtBeginSection[]{
  \begin{frame}
 \Huge \centerline{\insertsection}
  \small \tableofcontents[currentsection, hideothersubsections]
  \end{frame} 
}

%% TIKZ
\usepackage{tikz}
\usetikzlibrary{shapes.geometric, arrows}
\tikzstyle{startstop} = [rectangle, rounded corners, minimum width=3cm, minimum height=1cm,text centered, draw=black, fill=ulbeige!50]
\tikzstyle{io} = [trapezium, trapezium left angle=70, trapezium right angle=110, minimum width=3cm, minimum height=1cm, text centered, draw=black, fill=ulgray!50]
\tikzstyle{arrow} = [thick,->,>=stealth]
\usetikzlibrary{positioning}

\setcounter{tocdepth}{2}
\newcommand{\acc}[1]{\textcolor{ulred}{\textbf{#1}}}

\usepackage{tcolorbox}
