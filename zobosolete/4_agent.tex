

%%%%%%%%%%%%%%%%%%%%%%%%%%%%%%%%%%%%%%%%%%%%%%%%%%

%%%%%%%%%%%%%%%%%%%%%%%%%%%%%%%%%%%%%%%%%%%%%%%%%%
% \begin{frame}[label=ladila]{Model-building II: intelligence}
%  Evolutive pressure gives rise to the next leap, {\em intelligence}: agents that, starting from their static model (DNA in life)   build higher-level compressive models of the world within their lifetime, e.g., using brains.\vfill
 
%  Importantly, KT holds that both static-model and active-modeling agents enjoy structured experience, only that their level of structure is possibly different.  \vfill
 
 
%  [What comes after life and intelligence?]
 
% \end{frame}



%%%%%%%%%%%%%%%%%%%%%%%%%%%%%%%%%%%%%%%%%%%%%%%%%%
% \begin{frame}[label=ladila]{Model-building in artificial agents}
% As a consequence of the above, we should explore two routes to the construction of artificial model-building agents: \vfill

% A) {\bf Single generation} model building where agents are endowed with a {\em simplicity bias} (this is what we call the  {\em intelligence} approach). \vfill


% B) {\bf Transgenerational} model building ({\em life}) where the bias for simplicity  emerges naturally from the construction process that favors simple short programs  under evolutionary pressure in {\em  environments governed by simple laws}.  
% \end{frame}